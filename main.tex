%%%%%%%%%%%%%%%%%%%%%%%%%%%%%%%%%%%%%%%%%%%%%%%%%%%%%%%%%%%%%%%%
\begin{hidden}
 
* too much ``''s make the sentence look scattered and visually less recognizable. ``e.g.'' also.

* \em, \bf, \it are all obsolete \TeX primitives, and it does not take effect properly --- for example, {\bf {\it aaa}} shows ``aaa'' in italic but NOT IN BOLD. Use \emph{}, \textit{}, \textbf{} and so on.

* always use \ff, \fd, \cea, \pr, \mv , and do not use it directly, e.g. FF, FD/LAMA2011, etc. 

* use of footnotes should be minimized.

* IPC2011 should always be \ipc . The definition can later be modified in abbrev.sty .

* prefer separated words over hyphened words. domain
  independent>domain-independent, planner independent >
  planner-independent.

* Table, Figure, Fig., should not be used directly. Always use \refig and \reftbl. When the development flag is enabled, direct use of \ref signals an error.

* Caption ends with a period.
\end{hidden}

%%%%%%%%%%%%%%%%%%%%%%%%%%%%%%%%%%%%%%%%%%%%%%%%%%%%%%%%%%%%%%%%

\begin{abstract}
  Best-first search algorithms such as A* need to apply tie-breaking strategies in order to decide which node to expand when multiple search nodes have the same evaluation score.
 We investigate and improve tie-breaking strategies for
 cost-optimal search using A*.
%  We first experimentally analyze the performance of common tie-breaking
%  strategies that break ties according to the heuristic value of the
%  nodes.  We find that the tie-breaking strategy has a significant impact on search
%  algorithm performance when there are 0-cost operators that induce
%  large plateau regions in the search space. Based on this, we
%  develop two new classes of tie-breaking strategies.
% We  first propose a depth diversification strategy which breaks ties according to the distance from the entrance to the plateau, and then show that this new strategy 
% significantly outperforms standard
%  strategies on domains with 0-cost actions.
We propose a new framework for interpreting A* search as a series of satisficing searches within plateaus consisting of nodes with the same f-cost.
 Based on this framework, we investigate a second, new class of tie-breaking strategy, 
 a  multi-heuristic tie-breaking strategy
 which embeds inadmissible, distance-to-go variations of various heuristics within an admissible search.
 This is shown to further improve the performance
 in combination with the depth metric.
\end{abstract}

\section{Introduction}
\subsubparagraph{hidden topic}

BBB! \refig{fig:ip} \cite{Asai2016}

\begin{figure}[tb]
 \includegraphics{img/static/ip.png}
 \caption{This is Invasion Percolation}
 \label{fig:ip}
\end{figure}
